\documentclass[11pt]{article}

    \usepackage[breakable]{tcolorbox}
    \usepackage{parskip} % Stop auto-indenting (to mimic markdown behaviour)
    
    \usepackage{iftex}
    \ifPDFTeX
    	\usepackage[T1]{fontenc}
    	\usepackage{mathpazo}
    \else
    	\usepackage{fontspec}
    \fi

    % Basic figure setup, for now with no caption control since it's done
    % automatically by Pandoc (which extracts ![](path) syntax from Markdown).
    \usepackage{graphicx}
    % Maintain compatibility with old templates. Remove in nbconvert 6.0
    \let\Oldincludegraphics\includegraphics
    % Ensure that by default, figures have no caption (until we provide a
    % proper Figure object with a Caption API and a way to capture that
    % in the conversion process - todo).
    \usepackage{caption}
    \DeclareCaptionFormat{nocaption}{}
    \captionsetup{format=nocaption,aboveskip=0pt,belowskip=0pt}

    \usepackage[Export]{adjustbox} % Used to constrain images to a maximum size
    \adjustboxset{max size={0.9\linewidth}{0.9\paperheight}}
    \usepackage{float}
    \floatplacement{figure}{H} % forces figures to be placed at the correct location
    \usepackage{xcolor} % Allow colors to be defined
    \usepackage{enumerate} % Needed for markdown enumerations to work
    \usepackage{geometry} % Used to adjust the document margins
    \usepackage{amsmath} % Equations
    \usepackage{amssymb} % Equations
    \usepackage{textcomp} % defines textquotesingle
    % Hack from http://tex.stackexchange.com/a/47451/13684:
    \AtBeginDocument{%
        \def\PYZsq{\textquotesingle}% Upright quotes in Pygmentized code
    }
    \usepackage{upquote} % Upright quotes for verbatim code
    \usepackage{eurosym} % defines \euro
    \usepackage[mathletters]{ucs} % Extended unicode (utf-8) support
    \usepackage{fancyvrb} % verbatim replacement that allows latex
    \usepackage{grffile} % extends the file name processing of package graphics 
                         % to support a larger range
    \makeatletter % fix for grffile with XeLaTeX
    \def\Gread@@xetex#1{%
      \IfFileExists{"\Gin@base".bb}%
      {\Gread@eps{\Gin@base.bb}}%
      {\Gread@@xetex@aux#1}%
    }
    \makeatother

    % The hyperref package gives us a pdf with properly built
    % internal navigation ('pdf bookmarks' for the table of contents,
    % internal cross-reference links, web links for URLs, etc.)
    \usepackage{hyperref}
    % The default LaTeX title has an obnoxious amount of whitespace. By default,
    % titling removes some of it. It also provides customization options.
    \usepackage{titling}
    \usepackage{longtable} % longtable support required by pandoc >1.10
    \usepackage{booktabs}  % table support for pandoc > 1.12.2
    \usepackage[inline]{enumitem} % IRkernel/repr support (it uses the enumerate* environment)
    \usepackage[normalem]{ulem} % ulem is needed to support strikethroughs (\sout)
                                % normalem makes italics be italics, not underlines
    \usepackage{mathrsfs}
    

    
    % Colors for the hyperref package
    \definecolor{urlcolor}{rgb}{0,.145,.698}
    \definecolor{linkcolor}{rgb}{.71,0.21,0.01}
    \definecolor{citecolor}{rgb}{.12,.54,.11}

    % ANSI colors
    \definecolor{ansi-black}{HTML}{3E424D}
    \definecolor{ansi-black-intense}{HTML}{282C36}
    \definecolor{ansi-red}{HTML}{E75C58}
    \definecolor{ansi-red-intense}{HTML}{B22B31}
    \definecolor{ansi-green}{HTML}{00A250}
    \definecolor{ansi-green-intense}{HTML}{007427}
    \definecolor{ansi-yellow}{HTML}{DDB62B}
    \definecolor{ansi-yellow-intense}{HTML}{B27D12}
    \definecolor{ansi-blue}{HTML}{208FFB}
    \definecolor{ansi-blue-intense}{HTML}{0065CA}
    \definecolor{ansi-magenta}{HTML}{D160C4}
    \definecolor{ansi-magenta-intense}{HTML}{A03196}
    \definecolor{ansi-cyan}{HTML}{60C6C8}
    \definecolor{ansi-cyan-intense}{HTML}{258F8F}
    \definecolor{ansi-white}{HTML}{C5C1B4}
    \definecolor{ansi-white-intense}{HTML}{A1A6B2}
    \definecolor{ansi-default-inverse-fg}{HTML}{FFFFFF}
    \definecolor{ansi-default-inverse-bg}{HTML}{000000}

    % commands and environments needed by pandoc snippets
    % extracted from the output of `pandoc -s`
    \providecommand{\tightlist}{%
      \setlength{\itemsep}{0pt}\setlength{\parskip}{0pt}}
    \DefineVerbatimEnvironment{Highlighting}{Verbatim}{commandchars=\\\{\}}
    % Add ',fontsize=\small' for more characters per line
    \newenvironment{Shaded}{}{}
    \newcommand{\KeywordTok}[1]{\textcolor[rgb]{0.00,0.44,0.13}{\textbf{{#1}}}}
    \newcommand{\DataTypeTok}[1]{\textcolor[rgb]{0.56,0.13,0.00}{{#1}}}
    \newcommand{\DecValTok}[1]{\textcolor[rgb]{0.25,0.63,0.44}{{#1}}}
    \newcommand{\BaseNTok}[1]{\textcolor[rgb]{0.25,0.63,0.44}{{#1}}}
    \newcommand{\FloatTok}[1]{\textcolor[rgb]{0.25,0.63,0.44}{{#1}}}
    \newcommand{\CharTok}[1]{\textcolor[rgb]{0.25,0.44,0.63}{{#1}}}
    \newcommand{\StringTok}[1]{\textcolor[rgb]{0.25,0.44,0.63}{{#1}}}
    \newcommand{\CommentTok}[1]{\textcolor[rgb]{0.38,0.63,0.69}{\textit{{#1}}}}
    \newcommand{\OtherTok}[1]{\textcolor[rgb]{0.00,0.44,0.13}{{#1}}}
    \newcommand{\AlertTok}[1]{\textcolor[rgb]{1.00,0.00,0.00}{\textbf{{#1}}}}
    \newcommand{\FunctionTok}[1]{\textcolor[rgb]{0.02,0.16,0.49}{{#1}}}
    \newcommand{\RegionMarkerTok}[1]{{#1}}
    \newcommand{\ErrorTok}[1]{\textcolor[rgb]{1.00,0.00,0.00}{\textbf{{#1}}}}
    \newcommand{\NormalTok}[1]{{#1}}
    
    % Additional commands for more recent versions of Pandoc
    \newcommand{\ConstantTok}[1]{\textcolor[rgb]{0.53,0.00,0.00}{{#1}}}
    \newcommand{\SpecialCharTok}[1]{\textcolor[rgb]{0.25,0.44,0.63}{{#1}}}
    \newcommand{\VerbatimStringTok}[1]{\textcolor[rgb]{0.25,0.44,0.63}{{#1}}}
    \newcommand{\SpecialStringTok}[1]{\textcolor[rgb]{0.73,0.40,0.53}{{#1}}}
    \newcommand{\ImportTok}[1]{{#1}}
    \newcommand{\DocumentationTok}[1]{\textcolor[rgb]{0.73,0.13,0.13}{\textit{{#1}}}}
    \newcommand{\AnnotationTok}[1]{\textcolor[rgb]{0.38,0.63,0.69}{\textbf{\textit{{#1}}}}}
    \newcommand{\CommentVarTok}[1]{\textcolor[rgb]{0.38,0.63,0.69}{\textbf{\textit{{#1}}}}}
    \newcommand{\VariableTok}[1]{\textcolor[rgb]{0.10,0.09,0.49}{{#1}}}
    \newcommand{\ControlFlowTok}[1]{\textcolor[rgb]{0.00,0.44,0.13}{\textbf{{#1}}}}
    \newcommand{\OperatorTok}[1]{\textcolor[rgb]{0.40,0.40,0.40}{{#1}}}
    \newcommand{\BuiltInTok}[1]{{#1}}
    \newcommand{\ExtensionTok}[1]{{#1}}
    \newcommand{\PreprocessorTok}[1]{\textcolor[rgb]{0.74,0.48,0.00}{{#1}}}
    \newcommand{\AttributeTok}[1]{\textcolor[rgb]{0.49,0.56,0.16}{{#1}}}
    \newcommand{\InformationTok}[1]{\textcolor[rgb]{0.38,0.63,0.69}{\textbf{\textit{{#1}}}}}
    \newcommand{\WarningTok}[1]{\textcolor[rgb]{0.38,0.63,0.69}{\textbf{\textit{{#1}}}}}
    
    
    % Define a nice break command that doesn't care if a line doesn't already
    % exist.
    \def\br{\hspace*{\fill} \\* }
    % Math Jax compatibility definitions
    \def\gt{>}
    \def\lt{<}
    \let\Oldtex\TeX
    \let\Oldlatex\LaTeX
    \renewcommand{\TeX}{\textrm{\Oldtex}}
    \renewcommand{\LaTeX}{\textrm{\Oldlatex}}
    % Document parameters
    % Document title
    \title{ProjetoBigData\_GuilhermeSilvaMarineFournieGonçaloAlves}
    
    
    
    
    
% Pygments definitions
\makeatletter
\def\PY@reset{\let\PY@it=\relax \let\PY@bf=\relax%
    \let\PY@ul=\relax \let\PY@tc=\relax%
    \let\PY@bc=\relax \let\PY@ff=\relax}
\def\PY@tok#1{\csname PY@tok@#1\endcsname}
\def\PY@toks#1+{\ifx\relax#1\empty\else%
    \PY@tok{#1}\expandafter\PY@toks\fi}
\def\PY@do#1{\PY@bc{\PY@tc{\PY@ul{%
    \PY@it{\PY@bf{\PY@ff{#1}}}}}}}
\def\PY#1#2{\PY@reset\PY@toks#1+\relax+\PY@do{#2}}

\expandafter\def\csname PY@tok@w\endcsname{\def\PY@tc##1{\textcolor[rgb]{0.73,0.73,0.73}{##1}}}
\expandafter\def\csname PY@tok@c\endcsname{\let\PY@it=\textit\def\PY@tc##1{\textcolor[rgb]{0.25,0.50,0.50}{##1}}}
\expandafter\def\csname PY@tok@cp\endcsname{\def\PY@tc##1{\textcolor[rgb]{0.74,0.48,0.00}{##1}}}
\expandafter\def\csname PY@tok@k\endcsname{\let\PY@bf=\textbf\def\PY@tc##1{\textcolor[rgb]{0.00,0.50,0.00}{##1}}}
\expandafter\def\csname PY@tok@kp\endcsname{\def\PY@tc##1{\textcolor[rgb]{0.00,0.50,0.00}{##1}}}
\expandafter\def\csname PY@tok@kt\endcsname{\def\PY@tc##1{\textcolor[rgb]{0.69,0.00,0.25}{##1}}}
\expandafter\def\csname PY@tok@o\endcsname{\def\PY@tc##1{\textcolor[rgb]{0.40,0.40,0.40}{##1}}}
\expandafter\def\csname PY@tok@ow\endcsname{\let\PY@bf=\textbf\def\PY@tc##1{\textcolor[rgb]{0.67,0.13,1.00}{##1}}}
\expandafter\def\csname PY@tok@nb\endcsname{\def\PY@tc##1{\textcolor[rgb]{0.00,0.50,0.00}{##1}}}
\expandafter\def\csname PY@tok@nf\endcsname{\def\PY@tc##1{\textcolor[rgb]{0.00,0.00,1.00}{##1}}}
\expandafter\def\csname PY@tok@nc\endcsname{\let\PY@bf=\textbf\def\PY@tc##1{\textcolor[rgb]{0.00,0.00,1.00}{##1}}}
\expandafter\def\csname PY@tok@nn\endcsname{\let\PY@bf=\textbf\def\PY@tc##1{\textcolor[rgb]{0.00,0.00,1.00}{##1}}}
\expandafter\def\csname PY@tok@ne\endcsname{\let\PY@bf=\textbf\def\PY@tc##1{\textcolor[rgb]{0.82,0.25,0.23}{##1}}}
\expandafter\def\csname PY@tok@nv\endcsname{\def\PY@tc##1{\textcolor[rgb]{0.10,0.09,0.49}{##1}}}
\expandafter\def\csname PY@tok@no\endcsname{\def\PY@tc##1{\textcolor[rgb]{0.53,0.00,0.00}{##1}}}
\expandafter\def\csname PY@tok@nl\endcsname{\def\PY@tc##1{\textcolor[rgb]{0.63,0.63,0.00}{##1}}}
\expandafter\def\csname PY@tok@ni\endcsname{\let\PY@bf=\textbf\def\PY@tc##1{\textcolor[rgb]{0.60,0.60,0.60}{##1}}}
\expandafter\def\csname PY@tok@na\endcsname{\def\PY@tc##1{\textcolor[rgb]{0.49,0.56,0.16}{##1}}}
\expandafter\def\csname PY@tok@nt\endcsname{\let\PY@bf=\textbf\def\PY@tc##1{\textcolor[rgb]{0.00,0.50,0.00}{##1}}}
\expandafter\def\csname PY@tok@nd\endcsname{\def\PY@tc##1{\textcolor[rgb]{0.67,0.13,1.00}{##1}}}
\expandafter\def\csname PY@tok@s\endcsname{\def\PY@tc##1{\textcolor[rgb]{0.73,0.13,0.13}{##1}}}
\expandafter\def\csname PY@tok@sd\endcsname{\let\PY@it=\textit\def\PY@tc##1{\textcolor[rgb]{0.73,0.13,0.13}{##1}}}
\expandafter\def\csname PY@tok@si\endcsname{\let\PY@bf=\textbf\def\PY@tc##1{\textcolor[rgb]{0.73,0.40,0.53}{##1}}}
\expandafter\def\csname PY@tok@se\endcsname{\let\PY@bf=\textbf\def\PY@tc##1{\textcolor[rgb]{0.73,0.40,0.13}{##1}}}
\expandafter\def\csname PY@tok@sr\endcsname{\def\PY@tc##1{\textcolor[rgb]{0.73,0.40,0.53}{##1}}}
\expandafter\def\csname PY@tok@ss\endcsname{\def\PY@tc##1{\textcolor[rgb]{0.10,0.09,0.49}{##1}}}
\expandafter\def\csname PY@tok@sx\endcsname{\def\PY@tc##1{\textcolor[rgb]{0.00,0.50,0.00}{##1}}}
\expandafter\def\csname PY@tok@m\endcsname{\def\PY@tc##1{\textcolor[rgb]{0.40,0.40,0.40}{##1}}}
\expandafter\def\csname PY@tok@gh\endcsname{\let\PY@bf=\textbf\def\PY@tc##1{\textcolor[rgb]{0.00,0.00,0.50}{##1}}}
\expandafter\def\csname PY@tok@gu\endcsname{\let\PY@bf=\textbf\def\PY@tc##1{\textcolor[rgb]{0.50,0.00,0.50}{##1}}}
\expandafter\def\csname PY@tok@gd\endcsname{\def\PY@tc##1{\textcolor[rgb]{0.63,0.00,0.00}{##1}}}
\expandafter\def\csname PY@tok@gi\endcsname{\def\PY@tc##1{\textcolor[rgb]{0.00,0.63,0.00}{##1}}}
\expandafter\def\csname PY@tok@gr\endcsname{\def\PY@tc##1{\textcolor[rgb]{1.00,0.00,0.00}{##1}}}
\expandafter\def\csname PY@tok@ge\endcsname{\let\PY@it=\textit}
\expandafter\def\csname PY@tok@gs\endcsname{\let\PY@bf=\textbf}
\expandafter\def\csname PY@tok@gp\endcsname{\let\PY@bf=\textbf\def\PY@tc##1{\textcolor[rgb]{0.00,0.00,0.50}{##1}}}
\expandafter\def\csname PY@tok@go\endcsname{\def\PY@tc##1{\textcolor[rgb]{0.53,0.53,0.53}{##1}}}
\expandafter\def\csname PY@tok@gt\endcsname{\def\PY@tc##1{\textcolor[rgb]{0.00,0.27,0.87}{##1}}}
\expandafter\def\csname PY@tok@err\endcsname{\def\PY@bc##1{\setlength{\fboxsep}{0pt}\fcolorbox[rgb]{1.00,0.00,0.00}{1,1,1}{\strut ##1}}}
\expandafter\def\csname PY@tok@kc\endcsname{\let\PY@bf=\textbf\def\PY@tc##1{\textcolor[rgb]{0.00,0.50,0.00}{##1}}}
\expandafter\def\csname PY@tok@kd\endcsname{\let\PY@bf=\textbf\def\PY@tc##1{\textcolor[rgb]{0.00,0.50,0.00}{##1}}}
\expandafter\def\csname PY@tok@kn\endcsname{\let\PY@bf=\textbf\def\PY@tc##1{\textcolor[rgb]{0.00,0.50,0.00}{##1}}}
\expandafter\def\csname PY@tok@kr\endcsname{\let\PY@bf=\textbf\def\PY@tc##1{\textcolor[rgb]{0.00,0.50,0.00}{##1}}}
\expandafter\def\csname PY@tok@bp\endcsname{\def\PY@tc##1{\textcolor[rgb]{0.00,0.50,0.00}{##1}}}
\expandafter\def\csname PY@tok@fm\endcsname{\def\PY@tc##1{\textcolor[rgb]{0.00,0.00,1.00}{##1}}}
\expandafter\def\csname PY@tok@vc\endcsname{\def\PY@tc##1{\textcolor[rgb]{0.10,0.09,0.49}{##1}}}
\expandafter\def\csname PY@tok@vg\endcsname{\def\PY@tc##1{\textcolor[rgb]{0.10,0.09,0.49}{##1}}}
\expandafter\def\csname PY@tok@vi\endcsname{\def\PY@tc##1{\textcolor[rgb]{0.10,0.09,0.49}{##1}}}
\expandafter\def\csname PY@tok@vm\endcsname{\def\PY@tc##1{\textcolor[rgb]{0.10,0.09,0.49}{##1}}}
\expandafter\def\csname PY@tok@sa\endcsname{\def\PY@tc##1{\textcolor[rgb]{0.73,0.13,0.13}{##1}}}
\expandafter\def\csname PY@tok@sb\endcsname{\def\PY@tc##1{\textcolor[rgb]{0.73,0.13,0.13}{##1}}}
\expandafter\def\csname PY@tok@sc\endcsname{\def\PY@tc##1{\textcolor[rgb]{0.73,0.13,0.13}{##1}}}
\expandafter\def\csname PY@tok@dl\endcsname{\def\PY@tc##1{\textcolor[rgb]{0.73,0.13,0.13}{##1}}}
\expandafter\def\csname PY@tok@s2\endcsname{\def\PY@tc##1{\textcolor[rgb]{0.73,0.13,0.13}{##1}}}
\expandafter\def\csname PY@tok@sh\endcsname{\def\PY@tc##1{\textcolor[rgb]{0.73,0.13,0.13}{##1}}}
\expandafter\def\csname PY@tok@s1\endcsname{\def\PY@tc##1{\textcolor[rgb]{0.73,0.13,0.13}{##1}}}
\expandafter\def\csname PY@tok@mb\endcsname{\def\PY@tc##1{\textcolor[rgb]{0.40,0.40,0.40}{##1}}}
\expandafter\def\csname PY@tok@mf\endcsname{\def\PY@tc##1{\textcolor[rgb]{0.40,0.40,0.40}{##1}}}
\expandafter\def\csname PY@tok@mh\endcsname{\def\PY@tc##1{\textcolor[rgb]{0.40,0.40,0.40}{##1}}}
\expandafter\def\csname PY@tok@mi\endcsname{\def\PY@tc##1{\textcolor[rgb]{0.40,0.40,0.40}{##1}}}
\expandafter\def\csname PY@tok@il\endcsname{\def\PY@tc##1{\textcolor[rgb]{0.40,0.40,0.40}{##1}}}
\expandafter\def\csname PY@tok@mo\endcsname{\def\PY@tc##1{\textcolor[rgb]{0.40,0.40,0.40}{##1}}}
\expandafter\def\csname PY@tok@ch\endcsname{\let\PY@it=\textit\def\PY@tc##1{\textcolor[rgb]{0.25,0.50,0.50}{##1}}}
\expandafter\def\csname PY@tok@cm\endcsname{\let\PY@it=\textit\def\PY@tc##1{\textcolor[rgb]{0.25,0.50,0.50}{##1}}}
\expandafter\def\csname PY@tok@cpf\endcsname{\let\PY@it=\textit\def\PY@tc##1{\textcolor[rgb]{0.25,0.50,0.50}{##1}}}
\expandafter\def\csname PY@tok@c1\endcsname{\let\PY@it=\textit\def\PY@tc##1{\textcolor[rgb]{0.25,0.50,0.50}{##1}}}
\expandafter\def\csname PY@tok@cs\endcsname{\let\PY@it=\textit\def\PY@tc##1{\textcolor[rgb]{0.25,0.50,0.50}{##1}}}

\def\PYZbs{\char`\\}
\def\PYZus{\char`\_}
\def\PYZob{\char`\{}
\def\PYZcb{\char`\}}
\def\PYZca{\char`\^}
\def\PYZam{\char`\&}
\def\PYZlt{\char`\<}
\def\PYZgt{\char`\>}
\def\PYZsh{\char`\#}
\def\PYZpc{\char`\%}
\def\PYZdl{\char`\$}
\def\PYZhy{\char`\-}
\def\PYZsq{\char`\'}
\def\PYZdq{\char`\"}
\def\PYZti{\char`\~}
% for compatibility with earlier versions
\def\PYZat{@}
\def\PYZlb{[}
\def\PYZrb{]}
\makeatother


    % For linebreaks inside Verbatim environment from package fancyvrb. 
    \makeatletter
        \newbox\Wrappedcontinuationbox 
        \newbox\Wrappedvisiblespacebox 
        \newcommand*\Wrappedvisiblespace {\textcolor{red}{\textvisiblespace}} 
        \newcommand*\Wrappedcontinuationsymbol {\textcolor{red}{\llap{\tiny$\m@th\hookrightarrow$}}} 
        \newcommand*\Wrappedcontinuationindent {3ex } 
        \newcommand*\Wrappedafterbreak {\kern\Wrappedcontinuationindent\copy\Wrappedcontinuationbox} 
        % Take advantage of the already applied Pygments mark-up to insert 
        % potential linebreaks for TeX processing. 
        %        {, <, #, %, $, ' and ": go to next line. 
        %        _, }, ^, &, >, - and ~: stay at end of broken line. 
        % Use of \textquotesingle for straight quote. 
        \newcommand*\Wrappedbreaksatspecials {% 
            \def\PYGZus{\discretionary{\char`\_}{\Wrappedafterbreak}{\char`\_}}% 
            \def\PYGZob{\discretionary{}{\Wrappedafterbreak\char`\{}{\char`\{}}% 
            \def\PYGZcb{\discretionary{\char`\}}{\Wrappedafterbreak}{\char`\}}}% 
            \def\PYGZca{\discretionary{\char`\^}{\Wrappedafterbreak}{\char`\^}}% 
            \def\PYGZam{\discretionary{\char`\&}{\Wrappedafterbreak}{\char`\&}}% 
            \def\PYGZlt{\discretionary{}{\Wrappedafterbreak\char`\<}{\char`\<}}% 
            \def\PYGZgt{\discretionary{\char`\>}{\Wrappedafterbreak}{\char`\>}}% 
            \def\PYGZsh{\discretionary{}{\Wrappedafterbreak\char`\#}{\char`\#}}% 
            \def\PYGZpc{\discretionary{}{\Wrappedafterbreak\char`\%}{\char`\%}}% 
            \def\PYGZdl{\discretionary{}{\Wrappedafterbreak\char`\$}{\char`\$}}% 
            \def\PYGZhy{\discretionary{\char`\-}{\Wrappedafterbreak}{\char`\-}}% 
            \def\PYGZsq{\discretionary{}{\Wrappedafterbreak\textquotesingle}{\textquotesingle}}% 
            \def\PYGZdq{\discretionary{}{\Wrappedafterbreak\char`\"}{\char`\"}}% 
            \def\PYGZti{\discretionary{\char`\~}{\Wrappedafterbreak}{\char`\~}}% 
        } 
        % Some characters . , ; ? ! / are not pygmentized. 
        % This macro makes them "active" and they will insert potential linebreaks 
        \newcommand*\Wrappedbreaksatpunct {% 
            \lccode`\~`\.\lowercase{\def~}{\discretionary{\hbox{\char`\.}}{\Wrappedafterbreak}{\hbox{\char`\.}}}% 
            \lccode`\~`\,\lowercase{\def~}{\discretionary{\hbox{\char`\,}}{\Wrappedafterbreak}{\hbox{\char`\,}}}% 
            \lccode`\~`\;\lowercase{\def~}{\discretionary{\hbox{\char`\;}}{\Wrappedafterbreak}{\hbox{\char`\;}}}% 
            \lccode`\~`\:\lowercase{\def~}{\discretionary{\hbox{\char`\:}}{\Wrappedafterbreak}{\hbox{\char`\:}}}% 
            \lccode`\~`\?\lowercase{\def~}{\discretionary{\hbox{\char`\?}}{\Wrappedafterbreak}{\hbox{\char`\?}}}% 
            \lccode`\~`\!\lowercase{\def~}{\discretionary{\hbox{\char`\!}}{\Wrappedafterbreak}{\hbox{\char`\!}}}% 
            \lccode`\~`\/\lowercase{\def~}{\discretionary{\hbox{\char`\/}}{\Wrappedafterbreak}{\hbox{\char`\/}}}% 
            \catcode`\.\active
            \catcode`\,\active 
            \catcode`\;\active
            \catcode`\:\active
            \catcode`\?\active
            \catcode`\!\active
            \catcode`\/\active 
            \lccode`\~`\~ 	
        }
    \makeatother

    \let\OriginalVerbatim=\Verbatim
    \makeatletter
    \renewcommand{\Verbatim}[1][1]{%
        %\parskip\z@skip
        \sbox\Wrappedcontinuationbox {\Wrappedcontinuationsymbol}%
        \sbox\Wrappedvisiblespacebox {\FV@SetupFont\Wrappedvisiblespace}%
        \def\FancyVerbFormatLine ##1{\hsize\linewidth
            \vtop{\raggedright\hyphenpenalty\z@\exhyphenpenalty\z@
                \doublehyphendemerits\z@\finalhyphendemerits\z@
                \strut ##1\strut}%
        }%
        % If the linebreak is at a space, the latter will be displayed as visible
        % space at end of first line, and a continuation symbol starts next line.
        % Stretch/shrink are however usually zero for typewriter font.
        \def\FV@Space {%
            \nobreak\hskip\z@ plus\fontdimen3\font minus\fontdimen4\font
            \discretionary{\copy\Wrappedvisiblespacebox}{\Wrappedafterbreak}
            {\kern\fontdimen2\font}%
        }%
        
        % Allow breaks at special characters using \PYG... macros.
        \Wrappedbreaksatspecials
        % Breaks at punctuation characters . , ; ? ! and / need catcode=\active 	
        \OriginalVerbatim[#1,codes*=\Wrappedbreaksatpunct]%
    }
    \makeatother

    % Exact colors from NB
    \definecolor{incolor}{HTML}{303F9F}
    \definecolor{outcolor}{HTML}{D84315}
    \definecolor{cellborder}{HTML}{CFCFCF}
    \definecolor{cellbackground}{HTML}{F7F7F7}
    
    % prompt
    \makeatletter
    \newcommand{\boxspacing}{\kern\kvtcb@left@rule\kern\kvtcb@boxsep}
    \makeatother
    \newcommand{\prompt}[4]{
        \ttfamily\llap{{\color{#2}[#3]:\hspace{3pt}#4}}\vspace{-\baselineskip}
    }
    

    
    % Prevent overflowing lines due to hard-to-break entities
    \sloppy 
    % Setup hyperref package
    \hypersetup{
      breaklinks=true,  % so long urls are correctly broken across lines
      colorlinks=true,
      urlcolor=urlcolor,
      linkcolor=linkcolor,
      citecolor=citecolor,
      }
    % Slightly bigger margins than the latex defaults
    
    \geometry{verbose,tmargin=1in,bmargin=1in,lmargin=1in,rmargin=1in}
    
%%%%%%% PACKAGES (pode ser necessário acrescentar mais, dependendo do que se pretender inserir no documento) %%%%%%%
\usepackage[utf8]{inputenc}
\usepackage[portuges,portuguese]{babel} % para podermos escrever em português


% para que o índice possa ter o título de ``Índice'' (caso contrário fica ``Conteúdo'')
\addto\captionsportuguese{% Replace "english" with the language you use
  \renewcommand{\contentsname}%
    {Índice}%
}
\usepackage[nottoc,notlot,notlof]{tocbibind}

% para a inclusão de figuras
\usepackage{graphicx}

% para que não haja indentação no início dos parágrafos
\setlength{\parindent}{0pt} 

% para que os links apareçam como hiperligações
\usepackage{url}
\usepackage{hyperref}

\usepackage[usenames,dvipsnames]{color}    
%para introduzirmos fragmentos de script de R (ou de outra linguagem de programação)
\usepackage{listings} %para inserir excertos de codigo

\usepackage{subfigure} %Para incluir figuras lado a lado

\newcommand*{\authorimg}[1]{\raisebox{-.0\baselineskip}{\includegraphics[height=12pt,width=12pt,keepaspectratio,]{#1}}} %Para inserir o símbolo do R

\lstset{ 
  language=R,                     % linguagem
  basicstyle=\small\ttfamily, % tamanho das fontes usadas
  numbers=left,                   % onde colocar numeração das linhas de código
 numberstyle=\tiny\color{blue},  % estilo a usar para numeração das linhas
  stepnumber=1,                   % distância entre duas linhas numeradas (se for 1, cada linha será numerada)
  numbersep=5pt,                  % distância a que a numeração das linhas está do código
  backgroundcolor=\color{white},  % cor do background
  showspaces=false,               
  showstringspaces=false,         % sublinhar espaços em strings
  showtabs=false,                
  frame=single,                   % coloca uma moldura à volta do código
  rulecolor=\color{black},        % cor do frame
  tabsize=2,                    
  captionpos=b,                   % posição da legenda
  breaklines=true,                % line breaking automático
  breakatwhitespace=false,        
  keywordstyle=\color{RoyalBlue},      % estilo das keywords
  commentstyle=\color{YellowGreen},   % estilo dos comentários
  stringstyle=\color{ForestGreen}      % estilo das strings
} 


\begin{document}
    
\thispagestyle{empty}
% CAPA
\begin{flushleft}
\includegraphics[scale=0.2]{ESTB.jpg}
\end{flushleft}

\newline

\begin{center}
\Large{Instituto Politécnico de Setúbal}
\end{center}

\begin{center}
\Large{Escola Superior de Tecnologia do Barreiro}
\end{center}

\medskip % para dar um espaço vertical


\begin{center}
\Large{\textbf{Projeto de ``Big Data''}}
\end{center}
\begin{center}
\Large{Licenciatura em Bioinformática}
\end{center}

\vspace{3cm} % espaço vertical (uma alternativa ao \medskip, que pode ser customizada para efeitos estéticos)

\begin{center}
\huge{\textbf{Absenteísmo no Trabalho}} 
\end{center}


\begin{center}
\Large{Janeiro de 2023}
\end{center}

\vspace{2cm}

\medskip
\begin{center}

\large{Marine Fournier Nº202000224}

\large{Guilherme Silva Nº202000178}

\large{Gonçalo Alves Nº202000170}

\end{center}   

\pagebreak 

\newpage
\pagenumbering{roman}
%\phantomsection

% Página com o índice
\tableofcontents
\listoffigures

\newpage
\pagenumbering{arabic}


    \hypertarget{introduuxe7uxe3o}{%
\section{Introdução}\label{introduuxe7uxe3o}}

    \hypertarget{objetivos-e-fundamentos-do-projeto}{%
\subsection{Objetivos e Fundamentos do
projeto}\label{objetivos-e-fundamentos-do-projeto}}

    \begin{itemize}
\tightlist
\item
 Este projeto de BigData tem como objetivo usar algoritmos de Machine Learning, recorrendo ao uso do PySpark, pois o mesmo ajuda o processamento de grandes quantidades de dados e é incorpurado com ferramentas para prever resultados importantes em relação a um conjunto de dados.
 \item
O Spark é uma ferramenta poderosa que permite o processamento distribuído de dados.
Também vamos usar o Spark para ler, limpar e preparar os dados, avaliar e ajustar modelos e, finalmente, fazer previsões precisas e relevantes.
\item
Além disso, vamos usar técnicas avançadas de análise para visualizar e interpretar os resultados, a fim de obter insights valiosos.
\item
Este projeto é uma excelente oportunidade para aplicar e praticar habilidades de Machine Learning em um ambiente de grande escala e de alta performance.
\end{itemize}

    \hypertarget{introduuxe7uxe3o-do-dataset-em-uso}{%
\subsection{Introdução do DataSet em
uso}\label{introduuxe7uxe3o-do-dataset-em-uso}}

    \begin{itemize}
\tightlist

O nosso conjunto de dados inclui informações como as razões para ausência, gastos de transporte, distância de residência para trabalho, tempo de serviço, carga de trabalho, desempenho, educação, estado pessoal e informações demográficas dos funcionários, bem como a duração do absentismo.
\item
Temos como objetivo utilizar técnicas Machine Learning para identificar padrões e tendências nos dados e talvez prever a probabilidade de uma certa pessoa se absentir no futuro.
\item
Com base nas descobertas, pode-se tomar medidas para reduzir o absentismo e melhorar o desempenho de uma empresa.
\end{itemize}
\newline


\hypertarget{Criar uma Spark session}{%
\subsection{Criar uma Spark session}
\label{Criar uma Spark session}}
\begin{itemize}
\item
    Para se trabalhar com dados no Spark, é
    essencial criar uma SparkSession, este passo é fundamental e o primeiro
    a ser realizado. Ela é responsável por gerenciar a configuração do
    Spark, criar RDDs (Resilient Distributed Datasets) e DataFrames,
    registar tabelas temporárias e gerenciar os recursos do cluster
    utilizados.
\item
  A SparkSession é o ponto de partida para trabalhar com dados
  estruturados e relacionais no Spark SQL, o módulo do Spark que fornece
  suporte para essa funcionalidade. Ele permite que você execute
  consultas SQL, manipule DataFrames e extraia metadados. Ele também
  oferece acesso às bibliotecas de processamento de fluxo de dados (como
  o DataFrame API e o SQL) e às bibliotecas de aprendizado de máquina
  (como o MLlib) do Spark.
\item
  A SparkSession (criada usando SparkSession.builder) é uma classe
  fundamental para trabalhar com dados no Spark, é através dela que é
  possível configurar e gerenciar a sessão, estabelecer conexões com o
  cluster, definir opções de configuração e gerenciar recursos como o
  número de núcleos para usar e a quantidade de memória. É a porta de
  entrada para trabalhar com dados estruturados e relacionais no Spark
  SQL.
\item
  A configuração ``.config(''spark.memory.offHeap.enabled``)'' permite
  habilitar o uso de memória fora da heap (off-heap). Quando essa
  configuração é definida como ``true'', o Spark pode usar memória fora
  do heap do JVM para armazenar os dados e realizar operações. Isso
  permite que o Spark armazene e execute operações com mais dados, o que
  pode melhorar o desempenho quando a memória disponível dentro da heap
  não é suficiente.
\item
  A configuração ``.config(''spark.memory.offHeap.size``,''10g``)''
  permite definir a quantidade de memória fora do heap (off-heap)
  disponível para o Spark usar. Neste caso, o valor ``10g'' indica que o
  Spark pode usar até 10 gigabytes de memória fora do heap. Esse valor
  pode ser ajustado de acordo com o tamanho dos dados e as necessidades
  do seu aplicativo para garantir o melhor desempenho possível.
\end{itemize}

    \begin{tcolorbox}[breakable, size=fbox, boxrule=1pt, pad at break*=1mm,colback=cellbackground, colframe=cellborder]
\prompt{In}{incolor}{ }{\boxspacing}
\begin{Verbatim}[commandchars=\\\{\}]
\PY{k+kn}{from} \PY{n+nn}{pyspark}\PY{n+nn}{.}\PY{n+nn}{sql} \PY{k+kn}{import} \PY{n}{SparkSession}
\PY{k+kn}{import} \PY{n+nn}{findspark}

\PY{n}{findspark}\PY{o}{.}\PY{n}{init}\PY{p}{(}\PY{p}{)}

\PY{n}{spark} \PY{o}{=} \PY{n}{SparkSession}\PY{o}{.}\PY{n}{builder}\PY{o}{.}\PY{n}{appName}\PY{p}{(}\PY{l+s+s2}{\PYZdq{}}\PY{l+s+s2}{Pyspark ProjetoBD}\PY{l+s+s2}{\PYZdq{}}\PY{p}{)}\PYZbs{}
                    \PY{o}{.}\PY{n}{config}\PY{p}{(}\PY{l+s+s2}{\PYZdq{}}\PY{l+s+s2}{spark.memory.offHeap.enabled}\PY{l+s+s2}{\PYZdq{}}\PY{p}{,}\PY{l+s+s2}{\PYZdq{}}\PY{l+s+s2}{true}\PY{l+s+s2}{\PYZdq{}}\PY{p}{)}\PYZbs{}
                    \PY{o}{.}\PY{n}{config}\PY{p}{(}\PY{l+s+s2}{\PYZdq{}}\PY{l+s+s2}{spark.memory.offHeap.size}\PY{l+s+s2}{\PYZdq{}}\PY{p}{,}\PY{l+s+s2}{\PYZdq{}}\PY{l+s+s2}{20g}\PY{l+s+s2}{\PYZdq{}}\PY{p}{)}\PYZbs{}
                    \PY{o}{.}\PY{n}{getOrCreate}\PY{p}{(}\PY{p}{)}
\PY{n}{spark}
\end{Verbatim}
\end{tcolorbox}

            \begin{tcolorbox}[breakable, size=fbox, boxrule=.5pt, pad at break*=1mm, opacityfill=0]
\prompt{Out}{outcolor}{ }{\boxspacing}

\end{tcolorbox}

\textbf{''Data Visualization'' e Análise}

\begin{itemize}
\item
  Primeiramente vamos tamanho e o tipo de dados do nosso Dataset.
\item
  Este passo era mais simples ser executado através do método shape, mas
  como o não é suportado pelo PySpark DataFrame, no entanto, podemos
  contar o número de linhas e colunas usando o método count() e
  verificar o esquema (colunas e tipos de dados) do DataFrame usando o
  método printSchema().
\end{itemize}

    \begin{tcolorbox}[breakable, size=fbox, boxrule=1pt, pad at break*=1mm,colback=cellbackground, colframe=cellborder]
\prompt{In}{incolor}{ }{\boxspacing}
\begin{Verbatim}[commandchars=\\\{\}]
\PY{n+nb}{print}\PY{p}{(}\PY{l+s+s2}{\PYZdq{}}\PY{l+s+s2}{Numero de Linhas:}\PY{l+s+s2}{\PYZdq{}}\PY{p}{,} \PY{n}{df}\PY{o}{.}\PY{n}{count}\PY{p}{(}\PY{p}{)}\PY{p}{)}
\PY{n+nb}{print}\PY{p}{(}\PY{l+s+s2}{\PYZdq{}}\PY{l+s+s2}{Numero de Colunas:}\PY{l+s+s2}{\PYZdq{}}\PY{p}{,} \PY{n+nb}{len}\PY{p}{(}\PY{n}{df}\PY{o}{.}\PY{n}{columns}\PY{p}{)}\PY{p}{)}
\PY{n+nb}{print}\PY{p}{(}\PY{l+s+s2}{\PYZdq{}}\PY{l+s+s2}{Esquema do DataFrame: }\PY{l+s+s2}{\PYZdq{}}\PY{p}{)}
\PY{n}{df}\PY{o}{.}\PY{n}{printSchema}\PY{p}{(}\PY{p}{)}
\end{Verbatim}
\end{tcolorbox}

    \begin{Verbatim}[commandchars=\\\{\}]
Numero de Linhas: 1481
Numero de Colunas: 21
Esquema do DataFrame:
root
 |-- ID: string (nullable = true)
 |-- Reason for absence: string (nullable = true)
 |-- Month of absence: string (nullable = true)
 |-- Day of the week: string (nullable = true)
 |-- Seasons: string (nullable = true)
 |-- Transportation expense: string (nullable = true)
 |-- Distance from Residence to Work: string (nullable = true)
 |-- Service time: string (nullable = true)
 |-- Age: string (nullable = true)
 |-- Work load Average/day : string (nullable = true)
 |-- Hit target: string (nullable = true)
 |-- Disciplinary failure: string (nullable = true)
 |-- Education: string (nullable = true)
 |-- Son: string (nullable = true)
 |-- Social drinker: string (nullable = true)
 |-- Social smoker: string (nullable = true)
 |-- Pet: string (nullable = true)
 |-- Weight: string (nullable = true)
 |-- Height: string (nullable = true)
 |-- Body mass index: string (nullable = true)
 |-- Absenteeism time in hours: string (nullable = true)

    \end{Verbatim}

    Resumindo: - O DataFrame tem 8891 linhas e 21 colunas. O esquema do
DataFrame mostra que todas as colunas são do tipo string e são nullable
(podem conter valores nulos). Algumas das colunas incluem ``ID'',
``Reason for absence'', ``Month of absence'', ``Age'' e ``Absenteeism
time in hours''.

  
Descrição das variáveis do DataFrame:

\begin{itemize}
\tightlist
\item
  ID: Identificador único para cada registo.
\item
  Razão para a ausência: Razão para a ausência do funcionário no
  trabalho.
\item
  Mês da ausência: O mês em que o funcionário estava ausente.
\item
  Dia da semana: O dia da semana em que o funcionário estava ausente.
\item
  Estações: A estação do ano em que o funcionário estava ausente.
\item
  Despesas de transporte: As despesas de transporte incorridas pelo
  funcionário ao viajar para o trabalho.
\item
  Distância da residência para o trabalho: A distância entre a
  residência do funcionário e o local de trabalho.
\item
  Tempo de serviço: O período de tempo em que o funcionário está a
  trabalhar na empresa.
\item
  Idade: A idade do funcionário.
\item
  Carga de trabalho média/dia: A carga de trabalho média do funcionário
  por dia.
\item
  Alvo atingido: Se o funcionário atingiu ou não o alvo
\item
  Falha disciplinar: Se o funcionário cometeu alguma falha disciplinar
  ou não.
\item
  Educação: O nível de educação do funcionário.
\item
  Filhos: O número de filhos que o funcionário tem.
\item
  Bebedor social: Se o funcionário bebe socialmente ou não.
\item
  Fumador social: Se o funcionário fuma socialmente ou não.
\item
  Animal de estimação: Se o funcionário tem algum animal de estimação ou
  não.
\item
  Peso: O peso do funcionário.
\item
  Altura: A altura do funcionário.
\item
  Índice de massa corporal: O índice de massa corporal do funcionário.
\item
  Tempo de ausência em horas: O número de horas que o funcionário ficou
  ausente.
\end{itemize}

    \begin{tcolorbox}[breakable, size=fbox, boxrule=1pt, pad at break*=1mm,colback=cellbackground, colframe=cellborder]
\prompt{In}{incolor}{ }{\boxspacing}
\begin{Verbatim}[commandchars=\\\{\}]
\PY{n}{df} \PY{o}{=} \PY{n}{df}\PY{o}{.}\PY{n}{drop}\PY{p}{(}\PY{l+s+s2}{\PYZdq{}}\PY{l+s+s2}{id}\PY{l+s+s2}{\PYZdq{}}\PY{p}{)}
\end{Verbatim}
\end{tcolorbox}

    \begin{itemize}
\tightlist
\item
  Removeu-se a coluna ID pois não é relevante para os algoritmos de
  aprendizado de máquina.
\end{itemize}
\pagebreak
    Perguntas ao DataSet:

    1- Qual a dispersão dos pesos registados?

    \begin{tcolorbox}[breakable, size=fbox, boxrule=1pt, pad at break*=1mm,colback=cellbackground, colframe=cellborder]
\prompt{In}{incolor}{ }{\boxspacing}
\begin{Verbatim}[commandchars=\\\{\}]
\PY{k+kn}{from} \PY{n+nn}{pyspark}\PY{n+nn}{.}\PY{n+nn}{sql}\PY{n+nn}{.}\PY{n+nn}{functions} \PY{k+kn}{import} \PY{n}{stddev}

\PY{n}{df}\PY{o}{.}\PY{n}{groupBy}\PY{p}{(}\PY{p}{)}\PY{o}{.}\PY{n}{agg}\PY{p}{(}\PY{n}{stddev}\PY{p}{(}\PY{l+s+s2}{\PYZdq{}}\PY{l+s+s2}{Height}\PY{l+s+s2}{\PYZdq{}}\PY{p}{)}\PY{p}{)}\PY{o}{.}\PY{n}{show}\PY{p}{(}\PY{p}{)}
\end{Verbatim}
\end{tcolorbox}

    \begin{Verbatim}[commandchars=\\\{\}]
+-------------------+
|stddev\_samp(Height)|
+-------------------+
|  6.032953957248968|
+-------------------+

    \end{Verbatim}

    \begin{itemize}
\tightlist
\item
  Neste caso, podemos ver que os valores da coluna ``Height'' variam
  cerca de 6.03 unidades em relação à média.
\end{itemize}

2- Qual é a maior Distância do trabalho registada?

    \begin{tcolorbox}[breakable, size=fbox, boxrule=1pt, pad at break*=1mm,colback=cellbackground, colframe=cellborder]
\prompt{In}{incolor}{ }{\boxspacing}
\begin{Verbatim}[commandchars=\\\{\}]
\PY{n}{df}\PY{o}{.}\PY{n}{createOrReplaceTempView}\PY{p}{(}\PY{l+s+s2}{\PYZdq{}}\PY{l+s+s2}{data}\PY{l+s+s2}{\PYZdq{}}\PY{p}{)}
\PY{n}{spark}\PY{o}{.}\PY{n}{sql}\PY{p}{(}\PY{l+s+s2}{\PYZdq{}}\PY{l+s+s2}{SELECT MAX(`Distance from Residence to Work`) FROM data}\PY{l+s+s2}{\PYZdq{}}\PY{p}{)}\PY{o}{.}\PY{n}{show}\PY{p}{(}\PY{p}{)}
\end{Verbatim}
\end{tcolorbox}

    \begin{Verbatim}[commandchars=\\\{\}]
+------------------------------------+
|max(Distance from Residence to Work)|
+------------------------------------+
|  Distance from Res 52              |
+------------------------------------+

    \end{Verbatim}

    \begin{itemize}
\tightlist
\item
  A maior distância de residência para trabalho registada é de 52.
\end{itemize}

3- Qual é a soma da coluna ``Absenteeism time in hours'' para cada
valor único na coluna ``Education''?

    \begin{tcolorbox}[breakable, size=fbox, boxrule=1pt, pad at break*=1mm,colback=cellbackground, colframe=cellborder]
\prompt{In}{incolor}{ }{\boxspacing}
\begin{Verbatim}[commandchars=\\\{\}]
\PY{k+kn}{from} \PY{n+nn}{pyspark}\PY{n+nn}{.}\PY{n+nn}{sql}\PY{n+nn}{.}\PY{n+nn}{functions} \PY{k+kn}{import} \PY{n+nb}{sum}


\PY{n}{result} \PY{o}{=} \PY{n}{df}\PY{o}{.}\PY{n}{groupBy}\PY{p}{(}\PY{l+s+s2}{\PYZdq{}}\PY{l+s+s2}{Education}\PY{l+s+s2}{\PYZdq{}}\PY{p}{)}\PY{o}{.}\PY{n}{agg}\PY{p}{(}\PY{n+nb}{sum}\PY{p}{(}\PY{l+s+s2}{\PYZdq{}}\PY{l+s+s2}{Absenteeism time in hours}\PY{l+s+s2}{\PYZdq{}}\PY{p}{)}\PY{o}{.}\PY{n}{alias}\PY{p}{(}\PY{l+s+s2}{\PYZdq{}}\PY{l+s+s2}{sum\PYZus{}absenteeism}\PY{l+s+s2}{\PYZdq{}}\PY{p}{)}\PY{p}{)}
\PY{n}{result} \PY{o}{=} \PY{n}{result}\PY{o}{.}\PY{n}{orderBy}\PY{p}{(}\PY{l+s+s2}{\PYZdq{}}\PY{l+s+s2}{Education}\PY{l+s+s2}{\PYZdq{}}\PY{p}{)}
\PY{n}{result}\PY{o}{.}\PY{n}{show}\PY{p}{(}\PY{p}{)}
\end{Verbatim}
\end{tcolorbox}
    \begin{Verbatim}[commandchars=\\\{\}]
|Education|sum\_absenteeism|
+---------+---------------+
|        1|         8786.0|
|        2|          588.0|
|        3|          832.0|
|        4|           42.0|
|Education|           null|
+---------+---------------+

    \end{Verbatim}

    \begin{itemize}
\tightlist
\item
  Podemos ver que a educação de nível 3 tem uma soma de 416 horas de
  absenteísmo, a educação de nível 1 tem uma soma de 4393 horas de
  absenteísmo, a educação de nível 4 tem uma soma de 21 horas de
  absenteísmo e a educação de nível 2 tem uma soma de 294 horas de
  absenteísmo.
\item
  Ou seja, o Absenteeism time in hours praticamente diretamente
  relacionado com o nivel de Educação, as pessoas com educação de nivel
  1 ficam muito mais ausentes que os de nivel 4
\end{itemize}

Pré tratamento e Análise exploratória de dados

Vamos então verificar a existencia de valores ausentes em cada
coluna.

    \begin{tcolorbox}[breakable, size=fbox, boxrule=1pt, pad at break*=1mm,colback=cellbackground, colframe=cellborder]
\prompt{In}{incolor}{ }{\boxspacing}
\begin{Verbatim}[commandchars=\\\{\}]
\PY{c+c1}{\PYZsh{} criar uma lista para armazenar o número de valores ausentes em cada coluna.}
\PY{n}{missing\PYZus{}values} \PY{o}{=} \PY{p}{[}\PY{p}{]}

\PY{c+c1}{\PYZsh{} iterar através das colunas no conjunto de dados.}
\PY{k}{for} \PY{n}{col} \PY{o+ow}{in} \PY{n}{df}\PY{o}{.}\PY{n}{columns}\PY{p}{:}
    \PY{n}{missing\PYZus{}values}\PY{o}{.}\PY{n}{append}\PY{p}{(}\PY{p}{(}\PY{n}{col}\PY{p}{,} \PY{n}{df}\PY{o}{.}\PY{n}{filter}\PY{p}{(}\PY{n}{df}\PY{p}{[}\PY{n}{col}\PY{p}{]}\PY{o}{.}\PY{n}{isNull}\PY{p}{(}\PY{p}{)}\PY{p}{)}\PY{o}{.}\PY{n}{count}\PY{p}{(}\PY{p}{)}\PY{p}{)}\PY{p}{)}

\PY{c+c1}{\PYZsh{} print os valores inexistentes}
\PY{k}{for} \PY{n}{col}\PY{p}{,} \PY{n}{val} \PY{o+ow}{in} \PY{n}{missing\PYZus{}values}\PY{p}{:}
    \PY{k}{if} \PY{n}{val} \PY{o}{==} \PY{l+m+mi}{0}\PY{p}{:}
        \PY{n+nb}{print}\PY{p}{(}\PY{l+s+s2}{\PYZdq{}}\PY{l+s+si}{\PYZob{}\PYZcb{}}\PY{l+s+s2}{ : Nunhum missing values encontrado}\PY{l+s+s2}{\PYZdq{}}\PY{o}{.}\PY{n}{format}\PY{p}{(}\PY{n}{col}\PY{p}{)}\PY{p}{)}
    \PY{k}{else}\PY{p}{:}
        \PY{n+nb}{print}\PY{p}{(}\PY{l+s+s2}{\PYZdq{}}\PY{l+s+si}{\PYZob{}\PYZcb{}}\PY{l+s+s2}{ : }\PY{l+s+si}{\PYZob{}\PYZcb{}}\PY{l+s+s2}{ missing values}\PY{l+s+s2}{\PYZdq{}}\PY{o}{.}\PY{n}{format}\PY{p}{(}\PY{n}{col}\PY{p}{,} \PY{n}{val}\PY{p}{)}\PY{p}{)}
\end{Verbatim}
\end{tcolorbox}

    \begin{Verbatim}[commandchars=\\\{\}]
Reason for absence : Nunhum missing values encontrado
Month of absence : Nunhum missing values encontrado
Day of the week : Nunhum missing values encontrado
Seasons : Nunhum missing values encontrado
Transportation expense : Nunhum missing values encontrado
Distance from Residence to Work : Nunhum missing values encontrado
Service time : Nunhum missing values encontrado
Age : Nunhum missing values encontrado
Work load Average/day  : Nunhum missing values encontrado
Hit target : Nunhum missing values encontrado
Disciplinary failure : Nunhum missing values encontrado
Education : Nunhum missing values encontrado
Son : Nunhum missing values encontrado
Social drinker : Nunhum missing values encontrado
Social smoker : Nunhum missing values encontrado
Pet : Nunhum missing values encontrado
Weight : Nunhum missing values encontrado
Height : Nunhum missing values encontrado
Body mass index : Nunhum missing values encontrado
Absenteeism time in hours : Nunhum missing values encontrado
    \end{Verbatim}

    \begin{itemize}
\tightlist
\item
  Podemos então concluir que não existem missing values
\end{itemize}

    \hypertarget{transformauxe7uxe3o-dos-dados}{%
\subsection{Transformação dos
dados}\label{transformauxe7uxe3o-dos-dados}}

    \hypertarget{verifiquar-os-tipos-de-dados-das-colunas.}{%
\paragraph{Verifiquar os tipos de dados das
colunas.}\label{verifiquar-os-tipos-de-dados-das-colunas.}}

    \begin{tcolorbox}[breakable, size=fbox, boxrule=1pt, pad at break*=1mm,colback=cellbackground, colframe=cellborder]
\prompt{In}{incolor}{ }{\boxspacing}
\begin{Verbatim}[commandchars=\\\{\}]
\PY{n}{df}\PY{o}{.}\PY{n}{dtypes}
\end{Verbatim}
\end{tcolorbox}

            \begin{tcolorbox}[breakable, size=fbox, boxrule=.5pt, pad at break*=1mm, opacityfill=0]
\prompt{Out}{outcolor}{ }{\boxspacing}
\begin{Verbatim}[commandchars=\\\{\}]
[('Reason for absence', 'string'),
 ('Month of absence', 'string'),
 ('Day of the week', 'string'),
 ('Seasons', 'string'),
 ('Transportation expense', 'string'),
 ('Distance from Residence to Work', 'string'),
 ('Service time', 'string'),
 ('Age', 'string'),
 ('Work load Average/day ', 'string'),
 ('Hit target', 'string'),
 ('Disciplinary failure', 'string'),
 ('Education', 'string'),
 ('Son', 'string'),
 ('Social drinker', 'string'),
 ('Social smoker', 'string'),
 ('Pet', 'string'),
 ('Weight', 'string'),
 ('Height', 'string'),
 ('Body mass index', 'string'),
 ('Absenteeism time in hours', 'string')]
\end{Verbatim}
\end{tcolorbox}
        
    \begin{itemize}
\tightlist
\item
  É possível observar que todas as colunas são do tipo ``string'', o que
  pode precisar ser convertido para outro tipo de dado para melhorar a
  análise e modelagem de Machine Learning.
\end{itemize}

    \hypertarget{vamos-entuxe3o-converter-tipo-de-dados-string-para-double-para-melhorar-a-anuxe1lise-e-modelagem-de-machine-learning.}{%
\paragraph{Vamos então converter tipo de dados string para double para
melhorar a análise e modelagem de Machine
Learning.}\label{vamos-entuxe3o-converter-tipo-de-dados-string-para-double-para-melhorar-a-anuxe1lise-e-modelagem-de-machine-learning.}}

    \begin{tcolorbox}[breakable, size=fbox, boxrule=1pt, pad at break*=1mm,colback=cellbackground, colframe=cellborder]
\prompt{In}{incolor}{ }{\boxspacing}
\begin{Verbatim}[commandchars=\\\{\}]
\PY{k+kn}{from} \PY{n+nn}{pyspark}\PY{n+nn}{.}\PY{n+nn}{sql}\PY{n+nn}{.}\PY{n+nn}{functions} \PY{k+kn}{import} \PY{n}{col}
\PY{k}{for} \PY{n}{column} \PY{o+ow}{in} \PY{n}{df}\PY{o}{.}\PY{n}{columns}\PY{p}{:}
    \PY{n}{df} \PY{o}{=} \PY{n}{df}\PY{o}{.}\PY{n}{withColumn}\PY{p}{(}\PY{n}{column}\PY{p}{,} \PY{n}{col}\PY{p}{(}\PY{n}{column}\PY{p}{)}\PY{o}{.}\PY{n}{cast}\PY{p}{(}\PY{l+s+s2}{\PYZdq{}}\PY{l+s+s2}{double}\PY{l+s+s2}{\PYZdq{}}\PY{p}{)}\PY{p}{)}
\end{Verbatim}
\end{tcolorbox}

    \begin{tcolorbox}[breakable, size=fbox, boxrule=1pt, pad at break*=1mm,colback=cellbackground, colframe=cellborder]
\prompt{In}{incolor}{ }{\boxspacing}
\begin{Verbatim}[commandchars=\\\{\}]
\PY{n}{df}\PY{o}{.}\PY{n}{dtypes}
\end{Verbatim}
\end{tcolorbox}

            \begin{tcolorbox}[breakable, size=fbox, boxrule=.5pt, pad at break*=1mm, opacityfill=0]
\prompt{Out}{outcolor}{ }{\boxspacing}
\begin{Verbatim}[commandchars=\\\{\}]
[('Reason for absence', 'double'),
 ('Month of absence', 'double'),
 ('Day of the week', 'double'),
 ('Seasons', 'double'),
 ('Transportation expense', 'double'),
 ('Distance from Residence to Work', 'double'),
 ('Service time', 'double'),
 ('Age', 'double'),
 ('Work load Average/day ', 'double'),
 ('Hit target', 'double'),
 ('Disciplinary failure', 'double'),
 ('Education', 'double'),
 ('Son', 'double'),
 ('Social drinker', 'double'),
 ('Social smoker', 'double'),
 ('Pet', 'double'),
 ('Weight', 'double'),
 ('Height', 'double'),
 ('Body mass index', 'double'),
 ('Absenteeism time in hours', 'double')]
\end{Verbatim}
\end{tcolorbox}
        
    \begin{itemize}
\tightlist
\item
  Agora os dados podem ser manipulados e analisados com mais facilidade,
  pois o tipo ``double'' permite operações matemáticas, ao contrário do
  tipo ``string''. E também estão corretamente configurados para que os
  algoritmos de Machine Learning funcionem corretamente.
\end{itemize}

    \hypertarget{verifica-as-estatuxedsticas-das-colunas-numuxe9ricas.}{%
\subsection{As estatísticas das colunas
numéricas.}\label{verifica-as-estatuxedsticas-das-colunas-numuxe9ricas.}}

    \begin{tcolorbox}[breakable, size=fbox, boxrule=1pt, pad at break*=1mm,colback=cellbackground, colframe=cellborder]
\prompt{In}{incolor}{ }{\boxspacing}
\begin{Verbatim}[commandchars=\\\{\}]
\PY{n}{df}\PY{o}{.}\PY{n}{describe}\PY{p}{(}\PY{p}{)}\PY{o}{.}\PY{n}{toPandas}\PY{p}{(}\PY{p}{)}
\end{Verbatim}
\end{tcolorbox}


\begin{itemize}
\tightlist
\item
  É possível observar algumas informações gerais sobre cada coluna
  numérica no dataframe. Por exemplo, podemos ver o número total de
  entradas (count), a média, a desvio padrão, o valor mínimo e máximo.
  Estas informações podem ser úteis para entender melhor a distribuição
  dos dados e identificar possíveis outliers ou valores incomuns.
\end{itemize}

    \hypertarget{verifica-a-correlauxe7uxe3o-entre-as-colunas.}{%
\subsection{Verifica a correlação entre as
colunas.}\label{verifica-a-correlauxe7uxe3o-entre-as-colunas.}}

    \begin{tcolorbox}[breakable, size=fbox, boxrule=1pt, pad at break*=1mm,colback=cellbackground, colframe=cellborder]
\prompt{In}{incolor}{ }{\boxspacing}
\begin{Verbatim}[commandchars=\\\{\}]
\PY{n}{corr} \PY{o}{=} \PY{n}{df}\PY{o}{.}\PY{n}{stat}\PY{o}{.}\PY{n}{corr}\PY{p}{(}\PY{l+s+s2}{\PYZdq{}}\PY{l+s+s2}{Month of absence}\PY{l+s+s2}{\PYZdq{}}\PY{p}{,} \PY{l+s+s2}{\PYZdq{}}\PY{l+s+s2}{Transportation expense}\PY{l+s+s2}{\PYZdq{}}\PY{p}{)}
\PY{n+nb}{print}\PY{p}{(}\PY{n}{corr}\PY{p}{)}
\end{Verbatim}
\end{tcolorbox}

    \begin{Verbatim}[commandchars=\\\{\}]
0.1409565418102948
    \end{Verbatim}

    \begin{itemize}
\tightlist
\item
  A correlação varia entre -1 e 1, sendo que valores próximos a -1
  indicam uma correlação negativa forte, valores próximos a 1 indicam
  uma correlação positiva forte e valores próximos a 0 indicam ausência
  de correlação.
\item
  Neste caso como output \textasciitilde{} 0.1423, o que indica uma
  correlação fraca entre as duas colunas, não havendo realação entre os
  valores da coluna ``Month of absence'' e da coluna ``Transportation
  expense''.
\end{itemize}

    \hypertarget{verifica-o-nuxfamero-de-valores-uxfanicos-em-cada-coluna.}{%
\subsection{Verifica o número de valores únicos em cada
coluna.}\label{verifica-o-nuxfamero-de-valores-uxfanicos-em-cada-coluna.}}

    \begin{tcolorbox}[breakable, size=fbox, boxrule=1pt, pad at break*=1mm,colback=cellbackground, colframe=cellborder]
\prompt{In}{incolor}{ }{\boxspacing}
\begin{Verbatim}[commandchars=\\\{\}]
\PY{k}{for} \PY{n}{col} \PY{o+ow}{in} \PY{n}{df}\PY{o}{.}\PY{n}{columns}\PY{p}{:}
    \PY{n+nb}{print}\PY{p}{(}\PY{l+s+s2}{\PYZdq{}}\PY{l+s+s2}{Unique values in column }\PY{l+s+s2}{\PYZsq{}}\PY{l+s+si}{\PYZob{}\PYZcb{}}\PY{l+s+s2}{\PYZsq{}}\PY{l+s+s2}{:}\PY{l+s+s2}{\PYZdq{}}\PY{o}{.}\PY{n}{format}\PY{p}{(}\PY{n}{col}\PY{p}{)}\PY{p}{,} \PY{n}{df}\PY{o}{.}\PY{n}{select}\PY{p}{(}\PY{n}{col}\PY{p}{)}\PY{o}{.}\PY{n}{distinct}\PY{p}{(}\PY{p}{)}\PY{o}{.}\PY{n}{count}\PY{p}{(}\PY{p}{)}\PY{p}{)}
\end{Verbatim}
\end{tcolorbox}

    \begin{Verbatim}[commandchars=\\\{\}]
Unique values in column 'Reason for absence': 28
Unique values in column 'Month of absence': 14
Unique values in column 'Day of the week': 6
Unique values in column 'Seasons': 5
Unique values in column 'Transportation expense': 25
Unique values in column 'Distance from Residence to Work': 26
Unique values in column 'Service time': 19
Unique values in column 'Age': 23
Unique values in column 'Work load Average/day ': 39
Unique values in column 'Hit target': 14
Unique values in column 'Disciplinary failure': 3
Unique values in column 'Education': 5
Unique values in column 'Son': 6
Unique values in column 'Social drinker': 3
Unique values in column 'Social smoker': 3
Unique values in column 'Pet': 7
Unique values in column 'Weight': 27
Unique values in column 'Height': 15
Unique values in column 'Body mass index': 18
Unique values in column 'Absenteeism time in hours': 20
    \end{Verbatim}

    \begin{itemize}
\tightlist
\item
  Podemos então dizer que a coluna que tem mais valores unicos é a
  ``Reason for absence'' (28), enquanto que a coluna que tem menos é a
  ``Disciplinary failure'' (3).
\item
  Isto pode-nos ajudar a perceber a diversidade de valores em cada
  coluna e identificar se alguma coluna precisa ser tratada de forma
  diferente.
\end{itemize}

    \begin{itemize}
\tightlist
\item
  Com este output podemos ver o nome de cada coluna e o número de
  ocorrências para cada valor único. Isto permite uma visão geral dos
  dados, incluindo a verificação de valores ausentes (null) e a
  distribuição de valores para cada coluna.
\item
  A tabela ``Reason for absence'' apresenta a maior distribuição,
  seguida por ``Month of absence'' e ``Day of the week''. Os valores
  mais frequentes são ``Congenital malformations'', ``Diseases of the
  eye'' e ``Diseases of the musculoskeletal system''.
\end{itemize}

\pagebreak

\paragraph{Fig.1: Frequência de ausência de cada categoria de justificação.}

\begin{figure}[h]
   \includegraphics[scale=1.0]{ProjetoBigData_GuilhermeSilvaMarineFournieGonçaloAlves_63_0.png}
   \pagebreak
    \caption{Figura 1: Gráfico de barras que mostra a frequência de ausência de cada categoria de justificação}
    \pagebreak
   \label{fig:barras1}
\end{figure}

    \begin{itemize}
\tightlist
\item
  Esse gráfico mostra a frequência de ausência de cada categoria de
  justificação. A partir daí, é possível ver facilmente qual é a razão
  mais comum para a ausência que neste caso é ``Congenital
  malformations, deformations and chromosomal abnormalities'', e também
  comparar as frequências entre as diferentes razões. Também, pode-se
  observar que a maioria das razões tem baixa frequencia, e que poucas
  razões representam a maioria das ausencias.
\end{itemize}

    \hypertarget{gruxe1fico-que-mostra-a-relauxe7uxe3o-entre-transportation-expense-e-distance-from-residence-to-work}{%
\paragraph{Fig.2: Gráfico que mostra a relação entre ``Transportation expense''
e ``Distance from Residence to
Work''}\label{gruxe1fico-que-mostra-a-relauxe7uxe3o-entre-transportation-expense-e-distance-from-residence-to-work}}

\begin{figure}[h]
   \centering
   \includegraphics[scale=0.9]{ProjetoBigData_GuilhermeSilvaMarineFournieGonçaloAlves_66_0.png}
   \pagebreak
    \caption{Figura 2: Gráfico de dispersão que mostra a relação entre os
  gastos em transporte e a distância da residência para o trabalho.}
    \pagebreak
   \label{fig:dispresao}
\end{figure}

    \begin{itemize}
\tightlist
\item
  Este gráfico é um gráfico de dispersão que mostra a relação entre os
  gastos com transporte e a distância da residência para o trabalho. Ele
  parece estar mostrando que, em geral, quanto maior a distância da
  residência para o trabalho, maior os gastos com transporte. No
  entanto, há algumas exceções, como alguns pontos que apresentam gastos
  elevados com transporte mesmo com distâncias curtas. Este gráfico pode
  ser útil para entender se existe alguma relação entre essas duas
  variáveis e como elas podem afetar a saúde dos funcionários.
\end{itemize}


    \hypertarget{mostrar-a-distribuiuxe7uxe3o-de-idade-dos-funcionuxe1rios-que-estuxe3o-faltando}{%
\paragraph{Fig.3: Mostrar a distribuição de idade dos funcionários que estão
faltando}\label{mostrar-a-distribuiuxe7uxe3o-de-idade-dos-funcionuxe1rios-que-estuxe3o-faltando}}

\begin{figure}[h]
   \centering
   \includegraphics[scale=0.9]{ProjetoBigData_GuilhermeSilvaMarineFournieGonçaloAlves_69_0.png}
   \pagebreak
    \caption{Figura 3: Gráfico da distribuição de densidade dos
  funcionários ausentes em relação a idade}
    \pagebreak
   \label{fig:densidade}
\end{figure}
    
    \begin{itemize}
\tightlist
\item
  Este gráfico mostra a distribuição de densidade da idade dos
  funcionários ausentes. Ele mostra a frequência relativa de cada idade.
\item
  A grande maioria das pessoas ausentes estão na faixa etária entre 30 e
  40 anos.
\end{itemize}

\pagebreak

    \hypertarget{gruxe1ficos-histogramas-para-todas-as-colunas-numuxe9ricas-do-dataframe.}{%
\paragraph{Fig.4: Gráficos histogramas para todas as colunas numéricas do
dataframe.}\label{gruxe1ficos-histogramas-para-todas-as-colunas-numuxe9ricas-do-dataframe.}}

São apresentadas, de seguida, as distribuições dos valores das features.

\begin{figure}[h]
   \centering
   \includegraphics[scale=0.9]{ProjetoBigData_GuilhermeSilvaMarineFournieGonçaloAlves_73_0.png}
   \pagebreak
    \caption{Figura 4: Gráficos histogramas para todas as colunas numéricas do
dataframe.}
    \pagebreak
   \label{fig:histogramas}
\end{figure}
    
    \hypertarget{explicauxe7uxe3o-dos-histogramas}{%
\subparagraph{Explicação dos
histogramas:}\label{explicauxe7uxe3o-dos-histogramas}}

    \begin{itemize}
\tightlist
\item
  Estes histogramas permitem-nos ver a distribuição dos dados para cada
  coluna numérica e identificar tendências ou outliers. A partir do
  gráfico é possível ver a distribuição de frequencia para cada coluna
  numérica.
\item
  O eixo do x mostra os valores possíveis de cada feature, enquanto o
  eixo do y mostra a frequência com que esses valores aparecem no
  conjunto de dados, isto permite-nos identificar a distribuição dos
  dados e verificar se existem outliers ou valores incomuns.
\end{itemize}

    \hypertarget{anuxe1lise-dos-histogramas}{%
\subparagraph{Análise dos
histogramas:}\label{anuxe1lise-dos-histogramas}}

    \begin{itemize}
\tightlist
\item
  A coluna ``Age'' apresenta uma distribuição relativamente normal, com
  a maioria dos valores concentrados entre 20 e 40 anos de idade.
\item
  A coluna ``Work load Average/day'' apresenta valores concentrados
  entre cerca de 200 e 250, com poucos valores fora desse intervalo.
\item
  A coluna ``Hit target'' tem a maioria dos valores acima de 90,
  indicando que a maioria dos funcionários atingem suas metas.
\item
  A coluna ``Disciplinary failure'' tem a maioria dos valores em zero,
  indicando que a maioria dos funcionários não tem falhas disciplinares.
\item
  A coluna ``Education'' tem a maioria dos valores em 1, indicando que a
  maioria dos funcionários tem educação de nível básico.
\item
  A coluna ``Son'' tem valores distribuídos principalmente entre 0 e 4,
  indicando que a maioria dos funcionários tem entre 0 e 4 filhos.
\item
  A coluna ``Social drinker'' e ``Social smoker'' tem a maioria dos
  valores em 0, indicando que a maioria dos funcionários não são
  bebedores sociais ou fumantes sociais.
\item
  A coluna ``Pet'' tem valores distribuídos principalmente entre 0 e 2,
  indicando que a maioria dos funcionários tem entre 0 e 2 animais de
  estimação.
\item
  A coluna ``Weight'' tem valores distribuídos principalmente entre 60 e
  100, indicando que a maioria dos funcionários pesam entre 60 e 100 kg.
\item
  A coluna ``Height'' tem valores distribuídos principalmente entre 160
  e 180, indicando que a maioria dos funcionários tem entre 160 e 180 cm
  de altura.
\item
  A coluna ``Body mass index'' tem valores distribuídos principalmente
  entre 20 e 30, indicando que a maioria dos funcionários tem índice de
  massa corporal saudável.
\item
  A coluna ``Absenteeism time in hours'' tem valores distribuídos
  principalmente entre 0 e 20, indicando que a maioria dos funcionários
  tem entre 0 e 20 horas de ausência.
\end{itemize}
\pagebreak
    \hypertarget{gruxe1ficos-box-plot}{%
\subparagraph{Fig.5 e 6: Gráficos box plot}\label{gruxe1ficos-box-plot}}

\begin{figure}[h]
   \centering
   \includegraphics[scale=0.8]{ProjetoBigData_GuilhermeSilvaMarineFournieGonçaloAlves_79_3.png}
   \pagebreak
    \caption{Figura 5: Box plot "Transportation expense"}
    \pagebreak
   \label{fig:boxplot1}
\end{figure}

\begin{figure}[h]
   \centering
   \includegraphics[scale=0.8]{ProjetoBigData_GuilhermeSilvaMarineFournieGonçaloAlves_79_6.png}
   \pagebreak
    \caption{Figura 6: Box plot "Age"}
    \pagebreak
   \label{fig:boxplot2}
\end{figure}

    \begin{itemize}
\tightlist
\item
  Os gráficos box plot mostram a distribuição de dados numéricos e
  fornecem informações sobre a mediana, a variação e outliers dos dados.
  A ``caixa'' representa o intervalo interquartil (entre o primeiro e o
  terceiro quartil), ou seja, onde se encontra 50\% dos dados. A linha
  no meio da caixa representa a mediana (valor do meio). Os outliers são
  representados por pontos fora do intervalo de linhas.
\item
  Analisando supreficialmente alguns box plot podemos ver que no caso da
  coluna ``Age'', existe uma distribuição relativamente normal, com a
  maioria dos valores entre 20 e 40 anos, enquanto na coluna
  ``Transportation expense'' existem alguns valores outliers muito
  altos.
\end{itemize}

    \hypertarget{algoritmos-de-machine-learning}{%
\section{Algoritmos de Machine
Learning}\label{algoritmos-de-machine-learning}}

    \hypertarget{preparauxe7uxe3o-para-algoritmos-de-regressuxe3o}{%
\subsection{Preparação para Algoritmos de
Regressão}\label{preparauxe7uxe3o-para-algoritmos-de-regressuxe3o}}
        
    \begin{tcolorbox}[breakable, size=fbox, boxrule=1pt, pad at break*=1mm,colback=cellbackground, colframe=cellborder]
\prompt{In}{incolor}{ }{\boxspacing}
\begin{Verbatim}[commandchars=\\\{\}]
\PY{k+kn}{from} \PY{n+nn}{pyspark}\PY{n+nn}{.}\PY{n+nn}{ml}\PY{n+nn}{.}\PY{n+nn}{feature} \PY{k+kn}{import} \PY{n}{VectorAssembler}\PY{p}{,} \PY{n}{StandardScaler}
\PY{k+kn}{from} \PY{n+nn}{pyspark}\PY{n+nn}{.}\PY{n+nn}{ml}\PY{n+nn}{.}\PY{n+nn}{regression} \PY{k+kn}{import} \PY{n}{LinearRegression}

\PY{c+c1}{\PYZsh{} Selecione as colunas independentes}
\PY{n}{independent\PYZus{}cols} \PY{o}{=} \PY{p}{[}\PY{l+s+s1}{\PYZsq{}}\PY{l+s+s1}{Month of absence}\PY{l+s+s1}{\PYZsq{}}\PY{p}{,}
 \PY{l+s+s1}{\PYZsq{}}\PY{l+s+s1}{Day of the week}\PY{l+s+s1}{\PYZsq{}}\PY{p}{,}
 \PY{l+s+s1}{\PYZsq{}}\PY{l+s+s1}{Seasons}\PY{l+s+s1}{\PYZsq{}}\PY{p}{,}
 \PY{l+s+s1}{\PYZsq{}}\PY{l+s+s1}{Transportation expense}\PY{l+s+s1}{\PYZsq{}}\PY{p}{,}
 \PY{l+s+s1}{\PYZsq{}}\PY{l+s+s1}{Distance from Residence to Work}\PY{l+s+s1}{\PYZsq{}}\PY{p}{,}
 \PY{l+s+s1}{\PYZsq{}}\PY{l+s+s1}{Service time}\PY{l+s+s1}{\PYZsq{}}\PY{p}{,}
 \PY{l+s+s1}{\PYZsq{}}\PY{l+s+s1}{Age}\PY{l+s+s1}{\PYZsq{}}\PY{p}{,}
 \PY{l+s+s1}{\PYZsq{}}\PY{l+s+s1}{Work load Average/day }\PY{l+s+s1}{\PYZsq{}}\PY{p}{,}
 \PY{l+s+s1}{\PYZsq{}}\PY{l+s+s1}{Hit target}\PY{l+s+s1}{\PYZsq{}}\PY{p}{,}
 \PY{l+s+s1}{\PYZsq{}}\PY{l+s+s1}{Disciplinary failure}\PY{l+s+s1}{\PYZsq{}}\PY{p}{,}
 \PY{l+s+s1}{\PYZsq{}}\PY{l+s+s1}{Education}\PY{l+s+s1}{\PYZsq{}}\PY{p}{,}
 \PY{l+s+s1}{\PYZsq{}}\PY{l+s+s1}{Son}\PY{l+s+s1}{\PYZsq{}}\PY{p}{,}
 \PY{l+s+s1}{\PYZsq{}}\PY{l+s+s1}{Social drinker}\PY{l+s+s1}{\PYZsq{}}\PY{p}{,}
 \PY{l+s+s1}{\PYZsq{}}\PY{l+s+s1}{Social smoker}\PY{l+s+s1}{\PYZsq{}}\PY{p}{,}
 \PY{l+s+s1}{\PYZsq{}}\PY{l+s+s1}{Pet}\PY{l+s+s1}{\PYZsq{}}\PY{p}{,}
 \PY{l+s+s1}{\PYZsq{}}\PY{l+s+s1}{Weight}\PY{l+s+s1}{\PYZsq{}}\PY{p}{,}
 \PY{l+s+s1}{\PYZsq{}}\PY{l+s+s1}{Height}\PY{l+s+s1}{\PYZsq{}}\PY{p}{,}
 \PY{l+s+s1}{\PYZsq{}}\PY{l+s+s1}{Absenteeism time in hours}\PY{l+s+s1}{\PYZsq{}}\PY{p}{]}

\PY{c+c1}{\PYZsh{} Crie o objeto VectorAssembler}
\PY{n}{assembler} \PY{o}{=} \PY{n}{VectorAssembler}\PY{p}{(}\PY{n}{inputCols}\PY{o}{=}\PY{n}{independent\PYZus{}cols}\PY{p}{,} \PY{n}{outputCol}\PY{o}{=}\PY{l+s+s1}{\PYZsq{}}\PY{l+s+s1}{features}\PY{l+s+s1}{\PYZsq{}}\PY{p}{)}

\PY{c+c1}{\PYZsh{} Aplique o assembler no DataFrame}
\PY{n}{df\PYZus{}assembled} \PY{o}{=} \PY{n}{assembler}\PY{o}{.}\PY{n}{transform}\PY{p}{(}\PY{n}{df}\PY{p}{)}

\PY{c+c1}{\PYZsh{} Dividir os dados em conjunto de treinamento e teste}
\PY{n}{train}\PY{p}{,} \PY{n}{test} \PY{o}{=} \PY{n}{df\PYZus{}assembled}\PY{o}{.}\PY{n}{randomSplit}\PY{p}{(}\PY{p}{[}\PY{l+m+mf}{0.7}\PY{p}{,} \PY{l+m+mf}{0.3}\PY{p}{]}\PY{p}{)}
\end{Verbatim}
\end{tcolorbox}

    \begin{itemize}
\tightlist
\item
  Este passo está a selecionar as colunas independentes (colunas que
  serão usadas como variáveis ​​preditivas) para serem usadas na
  modelagem de regressão linear.
\item
  Criasse um objeto VectorAssembler, que é usado para combinar as
  colunas selecionadas em uma única coluna chamada ``features''. Este
  objeto é então aplicado ao DataFrame para produzir um novo DataFrame
  com uma coluna ``features'' que contém todas as variáveis
  ​​selecionadas.
\item
  Por fim, os dados são divididos em um conjunto de treinamento (70\%) e
  um conjunto de teste (30\%).
\end{itemize}

    \hypertarget{standarizar-os-dados-em-train-e-test}{%
\paragraph{Standarizar os dados em train e
test}\label{standarizar-os-dados-em-train-e-test}}

    \begin{tcolorbox}[breakable, size=fbox, boxrule=1pt, pad at break*=1mm,colback=cellbackground, colframe=cellborder]
\prompt{In}{incolor}{ }{\boxspacing}
\begin{Verbatim}[commandchars=\\\{\}]
\PY{c+c1}{\PYZsh{} Criar objeto StandardScaler}
\PY{n}{scaler} \PY{o}{=} \PY{n}{StandardScaler}\PY{p}{(}\PY{n}{inputCol}\PY{o}{=}\PY{l+s+s1}{\PYZsq{}}\PY{l+s+s1}{features}\PY{l+s+s1}{\PYZsq{}}\PY{p}{,} \PY{n}{outputCol}\PY{o}{=}\PY{l+s+s1}{\PYZsq{}}\PY{l+s+s1}{standard\PYZus{}features}\PY{l+s+s1}{\PYZsq{}}\PY{p}{)}

\PY{c+c1}{\PYZsh{} Aplique o scaler no conjunto de treinamento}
\PY{n}{scalerModel} \PY{o}{=} \PY{n}{scaler}\PY{o}{.}\PY{n}{fit}\PY{p}{(}\PY{n}{train}\PY{p}{)}

\PY{c+c1}{\PYZsh{} Aplique o scaler no conjunto de treinamento}
\PY{n}{train} \PY{o}{=} \PY{n}{scalerModel}\PY{o}{.}\PY{n}{transform}\PY{p}{(}\PY{n}{train}\PY{p}{)}

\PY{c+c1}{\PYZsh{} Aplique o scaler no conjunto de teste}
\PY{n}{test} \PY{o}{=} \PY{n}{scalerModel}\PY{o}{.}\PY{n}{transform}\PY{p}{(}\PY{n}{test}\PY{p}{)}
\end{Verbatim}
\end{tcolorbox}

    \begin{itemize}
\tightlist
\item
  Transforma as colunas `features' em `standard\_features', normalizando
  os valores dessas colunas para terem média 0 e desvio padrão 1.
\item
  Mostra os 2 primeiros registros das colunas `standard\_features' do
  conjunto de treinamento e teste, para que se você possa ver como os
  valores foram alterados.
\end{itemize}

    \hypertarget{k-means}{%
\subsection{K-means}\label{K-means}}
    \begin{itemize}
\tightlist
\item
K-means é um algoritmo fácil de entender e implementar, sendo muito utilizado em grandes conjuntos de dados.
\item
"Custo WCSS" é uma medida da similaridade entre os pontos de dados dentro de um cluster.
O objetivo é minimizar o WCSS, pois representa a soma das distâncias ao quadrado entre cada ponto de dados e o centroide de seu cluster.

\begin{figure}[h]
   \centering
   \includegraphics[scale=1.2]{kmeans02.png}
   \pagebreak
    \caption{Figura 7: K-means}
    \pagebreak
   \label{fig:kmeans02}
\end{figure}

\pagebreak
\item
O Elbow Method é uma técnica utilizada para determinar o número ótimo de clusters através do WCSS.
A ideia é que à medida que o número de clusters aumenta, o WCSS diminui, mas a dado momento esta diminuição pode não ser significativa para justificar a adição de mais clusters.
\item
Depois de analisar o gráfico Elbow, podemos concluir que o ponto ótimo para o número de clusters é k=6.
Isso significa que, a partir de 6 clusters, a taxa de variação do WCSS começa a diminuir significativamente, indicando que adicionar mais clusters não trará grandes benefícios em termos de agrupamento.
\end{itemize}


    \hypertarget{regressuxe3o-linear}{%
\subsection{Regressão Linear}\label{regressuxe3o-linear}}

    \begin{itemize}
\tightlist
\item
  A regressão linear é um dos algoritmos de aprendizado supervisionado
  mais simples e comuns utilizados para resolver problemas de previsão.
\item
  O objetivo é encontrar uma linha que melhor se ajuste aos dados, de
  forma a prever um valor de saída (variável dependente) com base em um
  ou mais valores de entrada (variáveis independentes).
\item
  A regressão linear é frequentemente utilizada para prever tendências
  futuras, identificar relações causais entre variáveis e para avaliar o
  impacto de uma ou mais variáveis independentes sobre uma variável
  dependente.
\item
  Em relação ao nosso DataSet é adequado usar Regressão Linear, pois
  podemos esperar uma relação direta entre as variáveis independentes e
  a variável dependente ``Body mass index''.
\end{itemize}
Elaboração do Modelo

    \begin{itemize}
\tightlist
\item
  Definição de cada variável (dependente e independente)
\item
  Treinar o modelo com os dados de treinamento. As previsões são feitas
  com os dados de teste usando o método ``transform'' do modelo
  treinado.
\end{itemize}


    \begin{Verbatim}[commandchars=\\\{\}]
Score:  0.6778241980124795
R2 = 0.994877
Root Mean Squared Error (RMSE) on test data = 0.322176
    \end{Verbatim}

    \begin{itemize}
\tightlist
\item
  Calculamos o Mean Squared Error (MSE), o Root Mean Squared Error
  (RMSE) e o score para avaliar a preformance do modelo.
\item
  A partir dos valores de score e rmse é possível concluir que o modelo
  tem uma precisão de 66.14\% e um erro médio quadrático de 0.33. Isso
  significa que o modelo tem uma boa capacidade de prever o índice de
  massa corporal dos funcionários, mas ainda tem um certo grau de
  incerteza.
\end{itemize}

    \hypertarget{matriz-de-confusuxe3o}{%
\paragraph{Matriz de Confusão?}\label{matriz-de-confusuxe3o}}

    \begin{itemize}
\tightlist
\item
  A matriz de confusão é uma tabela que mostra quantas previsões foram
  corretas e incorretas para cada classe. A diagonal principal contém as
  previsões corretas e as outras células contêm as previsões incorretas.
\item
  Criar uma matriz de confusão para um problema de regressão não faz
  sentido. A matriz de confusão é uma forma de avaliar o desempenho de
  um modelo de classificação, mas em problemas de regressão, as métricas
  de avaliação são diferentes, como Erro Quadrático Médio (MSE), Raiz do
  Erro Quadrático Médio (RMSE), Erro Absoluto Médio (MAE), entre outros.
\end{itemize}

    \hypertarget{gruxe1fico-de-regressuxe3o-linear}{%
\paragraph{Fig.8: Gráfico de Regressão
Linear}\label{gruxe1fico-de-regressuxe3o-linear}}

\begin{figure}[h]
   \centering
   \includegraphics[scale=0.9]{ProjetoBigData_GuilhermeSilvaMarineFournieGonçaloAlves_107_0.png}
   \pagebreak
    \caption{Figura 8: Gráfico de Regressão Linear}
    \pagebreak
   \label{fig:RegressãoLinear}
\end{figure}

    
    \begin{itemize}
\tightlist
\item
  Este gráfico mostra o resultado da regressão linear, onde cada ponto
  representa uma previsão feita pelo modelo para um determinado valor
  real.
\item
  Pode-se observar que a maioria dos pontos estão concentrados em torno
  da diagonal do gráfico, indicando que o modelo está fazendo previsões
  razoavelmente precisas.
\item
  Podemos ver a partir do gráfico que há uma certa correlação entre os
  valores previstos e os valores reais, mas também há alguns pontos que
  estão fora do normal, indicando que o modelo comete erros de previsão.
  Além disso, o valor de RMSE obtido foi de aproximadamente 0,34, o que
  é considerado um valor aceitável para um modelo de previsão. Portanto,
  podemos concluir que o modelo de regressão linear pode ser utilizado
  como um modelo de previsão de confiança, mas é importante considerar
  que ele pode cometer alguns erros.
\end{itemize}

    \hypertarget{uxe1rvore-de-regressuxe3o}{%
\subsection{Árvore de Regressão}\label{uxe1rvore-de-regressuxe3o}}

    \hypertarget{vantagens-do-uso-de-uxe1rvores-de-regressuxe3o}{%
\paragraph{Vantagens do uso de Árvores de
Regressão}\label{vantagens-do-uso-de-uxe1rvores-de-regressuxe3o}}

    \begin{itemize}
\tightlist
\item
  Facilidade de interpretação: Árvores de regressão são fáceis de
  entender e interpretar, pois mostram claramente as relações entre as
  variáveis de entrada e a variável de saída.
\item
  Não requer normalização: As árvores de regressão não requerem que os
  dados sejam normalizados antes do treinamento, o que é útil se os
  dados tiverem escalas diferentes.
\item
  Não afetado por outliers: As árvores de regressão são menos afetadas
  por outliers do que outros algoritmos de regressão, como a regressão
  linear.
\item
  Pode ser usado para problemas de classificação e regressão: As árvores
  de regressão podem ser usadas para resolver problemas tanto de
  classificação quanto de regressão.
\end{itemize}


    \begin{Verbatim}[commandchars=\\\{\}]
Score:  0.8481817366609425
R2 = 0.998862
Root Mean Squared Error (RMSE) on test data = 0.151818
    \end{Verbatim}

    \begin{itemize}
\tightlist
\item
  Pode-se concluir que o modelo tem uma alta precisão, pois o R² é
  próximo de 1, o que indica que a maioria dos dados estão perto da
  linha de tendência prevista pelo modelo.
\item
  O RMSE é relativamente baixo, indicando que a diferença entre os
  valores previstos e os valores reais são pequenas.
\item
  O Score mostra a porcentagem de acerto do modelo, no caso o valor é de
  84,8\%.
\end{itemize}


\begin{figure}[h]
   \centering
   \includegraphics[scale=0.9]{ProjetoBigData_GuilhermeSilvaMarineFournieGonçaloAlves_116_0.png}
   \pagebreak
    \caption{Figura 9: scatter plot entre os "Predicted values" e "Actuel Values" da Árvore de Regressão}
    \pagebreak
   \label{fig:scatter plot1}
\end{figure}

    
    \begin{itemize}
\tightlist
\item
  Este gráfico mostra uma dispersão entre os valores previstos pelo
  modelo (eixo x) e os valores reais (eixo y) para o índice de massa
  corporal.
\item
  Pode-se ver que a maioria dos pontos estão próximos da diagonal, o que
  indica que o modelo está fazendo previsões precisas.
\item
  No entanto, há alguns pontos que estão mal previstos, o que significa
  que o modelo está tendo dificuldade para prever esses valores.
\item
  O R2 = 0.998862 e o RMSE = 0.151818 o que significa que o modelo tem
  uma boa precisão.
\end{itemize}

    \hypertarget{random-forest}{%
\subsection{Random Forest}\label{random-forest}}

Porque usar Random Forest

    \begin{itemize}
\tightlist
\item
  Random Forest é um algoritmo de aprendizado de máquina supervisionado
  que pode ser utilizado tanto para classificação quanto para regressão.
\item
  Ele funciona criando várias árvores de decisão, e utilizando a média
  das previsões das árvores para fazer uma previsão final.
\item
  Isso ajuda a reduzir o overfitting, pois as árvores individuais tendem
  a se ajustar muito bem aos dados de treinamento, mas juntas elas
  funcionam de forma mais robusta.
\item
  Além disso, o Random Forest também tem a vantagem de ser capaz de
  lidar com múltiplas variáveis e features categóricas sem a necessidade
  de codificação.
\item
  Estas propriedades tornam o Random Forest uma escolha popular para
  muitos problemas de aprendizado de máquina.
\item
  Vamos então usar o Random forest para fazer uma previsão dos valores
  de ``Body mass index''
\end{itemize}
Elaboração do modelo
    \begin{itemize}
\tightlist
\item
  Criação do modelo de Random Forest para prever o índice de massa
  corporal (coluna ``Body mass index'') com base nas colunas numéricas
  (colunas ``numerical\_cols'') dos dados.
\item
  É então usado o objeto VectorAssembler para agrupar essas colunas em
  um único vetor chamado ``standard\_features''.
\item
  Os dados são então divididos em conjuntos de treinamento e teste (70\%
  treinamento e 30\% teste) e o modelo é treinado com os dados de
  treinamento.
\end{itemize}

    \begin{Verbatim}[commandchars=\\\{\}]
Score:  0.8325173179713301
R2 = 0.998374
Root Mean Squared Error (RMSE) on test data = 0.167483
    \end{Verbatim}

    \begin{itemize}
\tightlist
\item
  A partir do output apresentado, pode-se concluir que o modelo de
  random forest tem uma boa capacidade de previsão, pois o valor R² é
  próximo de 1 e indica que a maior parte da variação dos valores alvo é
  explicada pelo modelo.
\item
  Além disso, o erro médio quadrático raiz (RMSE) é relativamente baixo,
  o que também indica que as previsões são precisas.
\end{itemize}


\begin{figure}[h]
   \centering
   \includegraphics[scale=0.9]{ProjetoBigData_GuilhermeSilvaMarineFournieGonçaloAlves_125_0.png}
   \pagebreak
    \caption{Figura 10: Scatter plot entre os "Predicted values" e "Actuel Values" da Random Forest}
    \pagebreak
   \label{fig:scatterPlot2}
\end{figure}
    
    \begin{itemize}
\tightlist
\item
  Este gráfico apresenta uma dispersão dos valores previstos pelo modelo
  (eixo x) em relação aos valores reais (eixo y). Também sendo possível
  observar o quão precisas foram as previsões do modelo.
\item
  Uma boa precisão seria representada por pontos que seguem a diagonal,
  indicando que os valores previstos estão próximos dos valores reais.
\item
  Além disso, também é possível observar que o modelo apresenta alguns
  valores previstos maiores ou menores do que os valores reais.
\item
  O modelo é capaz de prever com precisão o índice de massa corporal, no
  entanto o mesmo apresenta algum erro, portanto seria necessário fazer
  mais análise e testes para determinar se o modelo é confiável o
  suficiente para ser usado em uma aplicação real.
\end{itemize}

    \hypertarget{conclusuxe3o}{%
\section{Conclusão}\label{conclusuxe3o}}

   \begin{itemize}
\tightlist
\item
    Algoritmos de Machine Learning automatizam a descoberta de padrões e tendências em dados em dum determinado DataSet com o objetivo de prever resultados futuros, identificar relações entre variáveis e tomar decisões relevantes.
    Além disso, os algoritmos de Machine Learning podem ser usados para melhorar a eficiência e precisão de processos existentes e para aumentar a capacidade de um sistema de lidar com grandes volumes de dados, como por exemplo o PySpark.
    \item
    O Spark foi usado para carregar, preparar e processar os dados, bem como para treinar e avaliar modelos de Aprendizagem Automática usando principalmente técnicas de regressão, como árvores de decisão e random forest. Além disso, de também foi usado para realizar consultas SQL.
    \item
    No inicio deste projeto foram utilizadas técnicas de processamento de dados, análise exploratória. Os dados foram limpos e preparados para o treinamento dos modelos de previsão.
    \item
    Depois de comparar todos os algoritmos de Machine Learning concluimos que todos os modelos apresentaram resultados razoáveis, mas o modelo de random forest obteve o melhor desempenho com um R2 de 0.998374 e um RMSE de 0.167483. A análise gráfica também confirmou estes resultados, apresentando uma boa relação entre os valores previstos e os valores reais.
    \item
    Em geral, pode-se concluir que este projeto foi capaz de prever o IMC (Variável usada em todos os algoritmos de previsão) de uma maenira eficaz.
    \item
    Foram encontradas algumas dificuldades ao longo deste projeto, tais como: Preparação dos dados (limpeza, manipulação e normalização dos dados), otimização de desempenho e interpretação dos resultados.
    \item
    Tivemos que também fazer algumas transformações dos dados, como a conversão de colunas de String para tipos numéricos, para que fossem adequados para a aplicação de algoritmos de regressão.
    \item
    Ao aplicarmos o algoritmo de clustering K-Means, conseguimos obter resultados satisfatórios, embora possa haver margem para melhoria. Analisando o gráfico de Elbow, identificamos que o ponto de inflexão ocorreu no valor k = 6, o que indica que talvez seja este o número ideal de clusters para este conjunto de dados.
\end{itemize}


%%% Bibliografia
\pagebreak
\begin{thebibliography}{}

\bibitem{Chambers_2019}
Chambers, B., & Zaharia, M. (2019). Spark: The Definitive Guide. O'Reilly Media, Inc.

\bibitem{Ryza_2019}
Ryza, S., Laserson, U., Owen, S., & Wills, J. (2019). Advanced Analytics with Spark. O'Reilly Media, Inc.

\bibitem{Raschka_2015}
Raschka, S., & Mirjalili, V. (2015). Python Machine Learning. Packt Publishing Ltd.

\bibitem{Pentreath_2015}
Pentreath, N. (2015). Machine Learning with Spark. Packt Publishing Ltd.

\bibitem{Thottuvaikkatumana_2019}
Thottuvaikkatumana, R. (2019). Spark for Python Developers. Packt Publishing Ltd.

\bibitem{Zinoviev_2017}
Zinoviev, D. (2017). Big Data Analysis with Python. Packt Publishing Ltd.

\bibitem{VanderPlas_2016}
VanderPlas, J. (2016). Python Data Science Handbook. O'Reilly Media, Inc.

\bibitem{Grus_2015}
Grus, J. (2015). Data Science from Scratch. O'Reilly Media, Inc.

\bibitem{Provost_2013}
Provost, F., & Fawcett, T. (2013). Data Science for Business. O'Reilly Media, Inc.

\bibitem{Muller_2016}
Müller, A., & Guido, S. (2016). Introduction to Machine Learning with Python. O'Reilly Media, Inc.

\end{thebibliography}
\end{document}
